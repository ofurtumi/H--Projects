\documentclass{article}
\usepackage[utf8]{inputenc}
\usepackage[icelandic]{babel}
\usepackage[a4paper, total={6in, 8in}]{geometry}
\usepackage[T1]{fontenc}
\usepackage{graphicx}
\usepackage{mathtools}
\usepackage{amsmath}
\usepackage{amssymb}
\usepackage{minted}


\graphicspath{ {./} }
\title{Vikublað 5 - Tölv-2}
\author{ttb3@hi.is}
\date{\today}


\begin{document}
\maketitle

\section*{2.3.4}
Byrjum með tölurnar $[1,2,3,4,5,6,7,8,9,10]$, fylkið er ekki sorterað þannig $1$ verður viðmiðsstak
\begin{align*}
    [1,\underline{2},3,4,5,6,7,8,9,\underline{10}]\\
    <--------\\
    \text{leitar alla leið að lo og finnur ekkert sem er minna en 2}\\
    [1,2,\underline{3},4,5,6,7,8,9,\underline{10}]\\
    <-------\\
    \text{þarf aftur að endurtaka sama skref nema núna er lo = 3}\\
    \text{þar sem tölunum er nú þegar raðað í röð má sjá að}\\
    \text{þetta endurtekur sig fyrir öll stök}\\
    [1,2,3,4,5,6,7,8,\underline{9},\underline{10}]\\
    <-\\
    \text{þetta er síðasta skrefið nú er leitað frá 10 að 9}\\
    \text{ekkert finnst og lykkjan brotnar}
\end{align*}
þá má skoða þessar aðgerðir útfrá fjölda staka athugum að fyrsta aðgerðin þar sem $lo=2$ og $hi=10$ eru framdar 9 samanburðir, þ.e. $(n-1)$ samanburðir.
fyrir næsta skref er $lo=3$ en $hi=10$ ennþá því það skipti aldrei um stað, núna eru framkvæmdir 8 samanburðir, $(n-2)$.
Þessi tala heldur áfram að minnka þangað til í síðasta skrefinu þar sem $lo=9$ og $hi=10$ og framkæmd er aðeins 1 samanburður, $n-9$.
Þá er fjöldi samanburða = $\sum_{i=1}^{n-1}(n-i)$ sem er einfaldlega jafnt og $45$ þetta gengur upp þar sem versti mögulegi tími er $\sim\frac{n^2}{2}$


Svipað dæmi um versta tíma væri með tölunum $[10,9,8,7,6,5,4,3,2,1]$,
hérna er í rauninni sama dæmi í gangi nema bara öfugt þannig ég tek bara eitt sýnidæmi.
Viðmiðsstak í þessu dæmi er 10 þar sem það er fremsta stakið.
\begin{align*}
    [10,\underline{9},8,7,6,5,4,3,2&,\underline{1}]\\
    ------>&\\
    \text{finnur strax stak sem er minna en 10}\\ 
    \text{en getur ekki fyrir sitt litla líf fundið stak sem er stærra en 10}
\end{align*}
Nú lendum við í sama veseni og undan, þ.e. að við þurfum að framkvæma $\sum_{i=1}^{n-1}(n-i)$ samanburði, og eins og við vitum eru það $45$ samanburðir sem eru $\sim\frac{n^2}{2}$ og uppfylla þannig skilyrðin.


\section*{2.3.24}
okok er ekki alveg viss um að ég skilji þetta en here goes.
Ef ég er með fylki með tölum frá $1-16$ raðað í röð, notum bara $[11,5,2,15,3,16,7,13,1,4,12,9,6,10,14,8]$, 
og $k=3$ þá þarf fyrst að velja af handahófi $2^3-1$ tölur til að fara í minna fylki.
Ég vel $7$ tölur: $[5,2,7,1,6,10,14]$ og raða þeim innbyrðis og fæ út $[1,2,5,6,7,10,14]$ svo nota ég $6$ sem skiptistak því það er í miðjunni eftir röðun.
Þá eftir að hafa notað $6$ sem skiptistak í upphaflega fylkinu fæ ég út $[5,2,3,1,4,\underline{6},11,15,16,7,13,12,9,10,14,8]$.
Næst tek ég miðjustakið í bæði vinstra og hægra hlutfylki skiptifylkisins og fæ vinstra megin $2$ og hægra megin $10$. 
Þau eru notuð sem skiptistök á upphaflega fylkið og ég fæ: $[1,\underline{2},5,3,4,6,11,15,16,7,13,12,9,10,14,8]$ og svo $[1,2,5,3,4,6,7,9,8,\underline{10},11,15,16,13,12,14]$,
Þetta er svo endurtekið einu sinni enn, eða fjórum sinnum eftir því hvernig þú horfir á það, þar sem $1$,$5$,$7$ og $14$ eru notuð sem skiptistök. 
Ég tek það saman í eitt og maður endar með: $[1,2,3,4,5,6,7,9,8,10,11,13,12,14,15,16]$ næstum alveg raðað fylki, einu sem eru off eru 9 og 8 og 13 og 12.
\end{document}