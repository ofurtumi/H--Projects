\documentclass{article}
\usepackage[utf8]{inputenc}
\usepackage[icelandic]{babel}
\usepackage[T1]{fontenc}
\usepackage{graphicx}
\usepackage{mathtools}
\usepackage{amsmath}
\usepackage{amssymb}
\usepackage{minted}


\graphicspath{ {./} }
\title{Heimadæmi 2 - Tölv 2}
\author{ttb3@hi.is}
\date{\today}

\begin{document}
\maketitle

\section*{1}
\subsection*{c)}
$\frac{(N+1)(N+2)}{N^2} => \frac{N²+3N+2}{N²} => \frac{N²}{N²} => 1$

\subsection*{d)}
$2N³$

\subsection*{e)}
$\frac{lg(2N)}{lg(N)} => \frac{lg(N)+lg(2)}{lg(N)} => \frac{lg(N)}{lg(N)} => 1$

\section*{2}
\subsection*{a)}
Línuleg þróun:

$f(N) = (N+\frac{N}{2}+\frac{N}{4}+\frac{N}{8}+...)$

\subsection*{b)}

$f(N) = (1+2+4+8+...+\frac{N}{2})$

\section*{4}
\subsection*{1. tilraun}
bjó fyrst til fylki af stærð N, sem var argument, með random tölum frá 0-100, fattaði svo að með engum mínustölum er bara hægt að mynda 0 með 0+0+0 og það eru ekki mjög hagnýtar upplýsingar
\begin{center}
    \begin{tabular}{|c|c|c|c|c|}
        \hline
        N&sums&time&ratio&lg ratio\\
        \hline
        100&0&0.007s&-&-\\
        \hline
        200&0&0.011s&1.54&0.65\\
        \hline
        400&0&0.054s&4.9&2.3\\
        \hline
        800&84&0.582s&10.77&3.43\\
        \hline
        1600&455&4.65s&8&3\\
        \hline
        3200&3276&37.15s&8&3\\
        \hline
    \end{tabular}
\end{center}

\subsection*{2. tilraun}
breytti því hvaða random tölur geta komið í fylkið, þær eru núna á bilinu -50 til og með 50
\begin{center}
    \begin{tabular}{|c|c|c|c|c|}
        \hline
        N&sums&time&ratio&lg ratio\\
        \hline
        100&1261&0.006s&-&-\\
        \hline
        200&10197&0.014s&2.33&1.2\\
        \hline
        400&78484&0.078s&5.6&2.5\\
        \hline
        800&631493&0.6s&7.7&2.9\\
        \hline
        1600&5024459&4.73s&8&3\\
        \hline
        3200&40456944&37.6s&8&3\\
        \hline
    \end{tabular}
\end{center}

\end{document}