\documentclass{article}
\usepackage[utf8]{inputenc}
\usepackage[icelandic]{babel}
\usepackage[T1]{fontenc}
\usepackage{graphicx}
\usepackage{mathtools}
\usepackage{amsmath}
\usepackage{amssymb}
\usepackage{minted}
\usepackage[dvipsnames]{xcolor}

\graphicspath{ {./} }
\title{Vikublað 4 - TÖL203G}
\author{ttb3@hi.is}
\date{\today}


\begin{document}
\maketitle


\section*{2.1.8}
Þar sem versti tími fyrir insertion sort er $O(n²)$ og besti tími, 
þar sem þarf ekkert að inserta, er tíminn $O(n)$, getur fylki þar sem stökum er raðað 
handahófskennt verið annaðhvort þeirra eða á milli, það skiptir ekki máli hversu margar tegundir 
af stökum eru til staðar í fylkinu

\section*{2.2.2}
\begin{center}
    $a[]$\\
    \begin{tabular}{|c|c|c|c|c|c|c|c|c|c|c|c|c|}
        \hline
        &0&1&2&3&4&5&6&7&8&9&10&11\\
        \hline
        &E&A&S&Y&Q&U&E&S&T&I&O&N\\
        \hline
        merge(a,0,1,1)&\textcolor{red}{A}&\textcolor{red}{E}&S&Y&Q&U&E&S&T&I&O&N\\
        \hline
        merge(a,2,2,3)&E&A&\textcolor{red}{S}&\textcolor{red}{Y}&Q&U&E&S&T&I&O&N\\
        \hline
        merge(a,0,1,3)&\textcolor{red}{A}&\textcolor{red}{E}&\textcolor{red}{S}&\textcolor{red}{Y}&Q&U&E&S&T&I&O&N\\
        \hline
        merge(a,4,4,5)&A&E&S&Y&\textcolor{red}{Q}&\textcolor{red}{U}&E&S&T&I&O&N\\
        \hline
        merge(a,6,6,7)&A&E&S&Y&Q&U&\textcolor{red}{E}&\textcolor{red}{S}&T&I&O&N\\
        \hline
        merge(a,4,5,7)&A&E&S&Y&\textcolor{red}{E}&\textcolor{red}{Q}&\textcolor{red}{S}&\textcolor{red}{U}&T&I&O&N\\
        \hline
        merge(a,0,3,7)&\textcolor{red}{A}&\textcolor{red}{E}&\textcolor{red}{E}&\textcolor{red}{Q}&\textcolor{red}{S}&\textcolor{red}{S}&\textcolor{red}{U}&\textcolor{red}{Y}&T&I&O&N\\
        \hline
        merge(a,8,8,9)&A&E&E&Q&S&S&U&Y&\textcolor{red}{I}&\textcolor{red}{T}&O&N\\
        \hline
        merge(a,8,8,9)&A&E&E&Q&S&S&U&Y&I&T&\textcolor{red}{N}&\textcolor{red}{O}\\
        \hline
        merge(a,10,10,11)&A&E&E&Q&S&S&U&Y&\textcolor{red}{I}&\textcolor{red}{N}&\textcolor{red}{O}&\textcolor{red}{T}\\
        \hline
        merge(a,0,7,11)&\textcolor{red}{A}&\textcolor{red}{E}&\textcolor{red}{E}&\textcolor{red}{I}&\textcolor{red}{N}&\textcolor{red}{O}&\textcolor{red}{Q}&\textcolor{red}{S}&\textcolor{red}{S}&\textcolor{red}{T}&\textcolor{red}{U}&\textcolor{red}{Y}\\
        \hline
    \end{tabular}
\end{center}

\section*{2.2.22}
fyrir "venjulegt" merge sort, þar sem skipt er í tvennt í byrjun, 
er tíminn $O(n\log_2(n))$, 
þegar skipt er í þrennt breytist $\log_2$ yfir í $\log_3$
Tímaflækjan verður þannig $O(n\log_3(n))$.

\end{document}