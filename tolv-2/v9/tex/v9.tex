\documentclass{article}
\usepackage[utf8]{inputenc}
\usepackage[icelandic]{babel}
\usepackage[T1]{fontenc}
\usepackage{graphicx}
\usepackage{mathtools}
\usepackage{amsmath}
\usepackage{amssymb}
\usepackage{minted}
\usepackage{listings}
\usepackage{color}
\usepackage{caption}

\definecolor{dkgreen}{rgb}{0,0.6,0}
\definecolor{gray}{rgb}{0.5,0.5,0.5}
\definecolor{mauve}{rgb}{0.58,0,0.82}

\lstset{frame=tb,
  language=Python,
  aboveskip=3mm,
  belowskip=3mm,
  showstringspaces=false,
  columns=flexible,
  basicstyle={\small\ttfamily},
  numbers=none,
  numberstyle=\tiny\color{gray},
  keywordstyle=\color{blue},
  commentstyle=\color{dkgreen},
  stringstyle=\color{mauve},
  breaklines=true,
  breakatwhitespace=true,
  tabsize=3
}


\graphicspath{ {./} }
\title{Verkefni 9 - Tölvunarfræði 2}
\author{ttb3@hi.is}
\date{\today}


\begin{document}
\maketitle


\section*{1. 3.4.4}
skrifaði smá python kóða til að bruteforcea þetta dæmi, prófar fyrstu milljón a fyrir hvert og eitt M,
prentar a og M þegar stökin eru 10 mismunandi
\begin{lstlisting}
    tolur = [83,69,65,82,67,72,88,77,80,76]

    outTolur = []
    M = 9
    while (True):
        M+=1
        print(M)
        for a in range(1,1000000):
            for k in tolur:
                outTolur.append((a*k)%M)
            if(len(set(outTolur))==10): break
            outTolur = []
        if(len(set(outTolur))==10): break

    print('outSet',set(outTolur))
    print('outTolur',outTolur)
    print('a',a)
    print('M',M)
\end{lstlisting}
a = 1 og M = 20
þá er output $[3, 9, 5, 2, 7, 12, 8, 17, 0, 16]$

\newpage
\section*{2.}
Allar aðgerðir framkvæmdar 10 sinnum
\subsection*{BST}
\begin{tabular}{|l|c|c|}
    \hline
    case&insertion time (s)&search time (s)\\
    \hline
    worst&0.211&0.324\\
    \hline
    best&0.185&0.218\\
    \hline
    avg.&0.197&0.274\\
    \hline
\end{tabular}

\subsection*{RedBlackBST}
\begin{tabular}{|l|c|c|}
    \hline
    case&insertion time (s)&search time (s)\\
    \hline
    worst&0.230&0.356\\
    \hline
    best&0.203&0.221\\
    \hline
    avg.&0.217&0.254\\
    \hline
\end{tabular}

\subsection*{SCHST}
\begin{tabular}{|l|c|c|}
    \hline
    case&insertion time (s)&search time (s)\\
    \hline
    worst&0.197&0.243\\
    \hline
    best&0.170&0.219\\
    \hline
    avg.&0.181&0.227\\
    \hline
\end{tabular}

\subsection*{LPHST}
\begin{tabular}{|l|c|c|}
    \hline
    case&insertion time (s)&search time (s)\\
    \hline
    worst&0.201&0.273\\
    \hline
    best&0.147&0.219\\
    \hline
    avg.&0.160&0.236\\
    \hline
\end{tabular}


\lstset{frame=tb,
  language=java,
  aboveskip=3mm,
  belowskip=3mm,
  showstringspaces=false,
  columns=flexible,
  basicstyle={\small\ttfamily},
  numbers=none,
  numberstyle=\tiny\color{gray},
  keywordstyle=\color{blue},
  commentstyle=\color{dkgreen},
  stringstyle=\color{mauve},
  breaklines=true,
  breakatwhitespace=true,
  tabsize=3
}
\section*{3. 3.4.31}

\subsection*{put}
\begin{lstlisting}
    ...
    Key tempKey = null;
        Value tempVal = null;

        int keyH1 = key.hashCode();

        if (keys[0][keyH1] != null) {
            tempKey = keys[0][keyH1];
            tempVal = vals[0][keyH1];
            vals[0][keyH1] = val;
        } else {
            keys[0][keyH1] = key;
            vals[0][keyH1] = val;
        }

        if ((tempKey != null) && (keys[1][tempKey.hashCode()] != null)) {
            vals[1][tempKey.hashCode()] = tempVal;
            Key tTempKey = keys[1][tempKey.hashCode()];
            Value tTempVal = vals[1][tempKey.hashCode()];
            put(tTempKey, tTempVal);
        }
        else {
            keys[1][tempKey.hashCode()] = tempKey;
            vals[1][tempKey.hashCode()] = tempVal;
        }
    }
\end{lstlisting}

\subsection*{get}
\begin{lstlisting}
    public Value get(Key key) {
        if (key == null)
            throw new IllegalArgumentException("argument to get() is null");
        long h = murmurhash(key.hashCode(), 42);
        int h1 = ((int) (h & 0x7fffffff)) % m;
        int h2 = ((int) ((h >> 32) & 0x7fffffff)) % m;
        if (h > h1) return vals[1][(int)h];
        else return vals[0][(int)h];
    }
\end{lstlisting}

\section*{4. 4.1.4}
Til að finna út hvort vegur liggi á milli tveggja hnúta í óstefndu neti þarf bara að athuga hvort það sé tenging á milli þeirra.
Graph hefur poka af pokum sem halda utan um tengingar hvers og eins hnútar. 
Þar sem þessi gögn eru auðveldlega sótt er málið einfaldlega að tékka á hvort hnútur V sem er tekinn inn hafi hnút W í pokanum sínum.

\begin{lstlisting}
    public boolean hasEdge(int V, int W) {
        for (int node : adj(v)) {
            if (node.equals(w)) return true;
        }
        return false;
    }
\end{lstlisting}

\end{document}