\documentclass{article}
\usepackage[utf8]{inputenc}
\usepackage[icelandic]{babel}
\usepackage[T1]{fontenc}
\usepackage{graphicx}
\usepackage{mathtools}
\usepackage{amsmath}
\usepackage{amssymb}
\usepackage{minted}


\graphicspath{ {./} }
\title{Vikublað 7 - Tölv 2}
\author{ttb3@hi.is}
\date{\today}


\begin{document}
\maketitle


\section*{3.2.4}
\subsection*{a)}
a tréð er gefið sem $[10,9,8,7,6,5,4]$

a gengur upp, ef maður fylgir trénu sést að $10 > 9 > 8 > ... > 5$, þannig maður fylgir bara vinstri greinum trésins þangað til að maður lendir á 5

\subsection*{b)}
b tréð er gefið sem $[4,10,8,7,5,3]$

nú byrjar maður á $4$, 
þar sem $4<5$ er eina leiðin til að finna $5$ að fara til hægri, 
það er gert þegar farið er á $10$, 
næsta tékk er $10>8$ þannig farið er til vinstri. 
Eins og er gengur þetta upp,
haldið er áfram að fara eftir vinstri hlið trésins þangað til að komið er að $5$,
næst í röðinni á eftir $5$ er $3$ en það gengur ekki upp því $3<4$ og ætti þessvegna að vera á vinstri grein frá rótinni og ekki fyrir neðan $5$.
Þá má sjá að b er röð aðferða sem gengur ekki upp.

\subsection*{c)}
c tréð er gefið sem $[1,10,2,9,3,8,4,7,6,5]$

hérna skilar röð aðferða tré sem sikksakkar, byrjar á hægri,vinstri,hægri
þangað til að komið er að $6$ þá fer maður aftur til vinstri og lendir á $5$,
rétt gildi fundið, allir glaðir :)

\subsection*{d)}
d tréð er gefið sem $[2,7,3,8,4,5]$

hérna er tré sem ekki gengur upp, það fer frá $2$ til hægri í $7$ og frá $7$
vinstri til $3$ en svo hættir gamanið því það ætti að fara frá $3$ til hægri á $8$
nema hvað að $8>7$ þannig það ætti ekki að geta átt heima í vinstra hluttré $7$

d virkar ekki

\subsection*{e)}
e tréð er gefið sem $[1,2,10,4,8,5]$

röð aðferða verður hægri, hægri, vinstri, hægri, vinstri.
það gengur upp

\section*{3.2.15}
ritháttur:\\
$Y\rightarrow X$ = fara til hægri frá Y og lenda á X \\$Y\leftarrow X$ = fara til vinstri frá Y og lenda á X
\subsection*{a)}
floor("Q"), leitar að Q og skilar minnsta stakinu sem finnst á leiðinni\\
$E>Q:E\rightarrow Q$ þ.e. vegna þess að E er stærra en Q, fara frá E til hægri og lenda á Q. 
þar sem að Q er það sem leitað var að, skilar Q

\end{document}