\documentclass{article}
\usepackage[utf8]{inputenc}
\usepackage[icelandic]{babel}
\usepackage[T1]{fontenc}
\usepackage{graphicx}
\usepackage{mathtools}
\usepackage{amsmath}
\usepackage{amssymb}
\usepackage{minted}


\graphicspath{ {./} }
\title{Titill - Áfangi}
\author{ttb3@hi.is}
\date{\today}


\begin{document}
\maketitle 

\section*{2.4.5}
Ef maður setur inn lyklanna EASYQUESTION í tómt heap endar maður með YTUSQNSAEIOE. Tréð myndi líta (nokkurn vegin) svona út:
\begin{align}
    Y&\\
    T\quad&U\\
    S\quad Q\quad &N\quad S\\
    A\quad E\quad I\quad O\quad& E
\end{align}

\section*{2.4.10}

Gamla númeringin er barn(i,0) = 2xi (i=1,...N), barn(i,1) = 2xi+1,foreldri(i)=i/2
f(i)=i-1 # frá gömlu númerum yfir í nýju
foreldri_nýja(j) = foreldri(j+1) = (j+i)/2  , j = 0...N-1

þannig svarið er að foreldri pq[k] væri (k+1)/2

\end{document}